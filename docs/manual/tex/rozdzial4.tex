	\newpage
\section{Implementacja}		%4
%Opisać implementacje algorytmu/programu. Pokazać ciekawe fragmenty kodu
%Opisać powstałe wyniki (algorytmu/nrzędzia)

\subsection{Klasa Matrix} \label{sec:Matrix}

Klasa jest zaimplementowana jako jeden plik \texttt{.hpp}. Nie jest podzielona na plik implementacji oraz nagłówek, ponieważ jest ona szablonem. Deklaracja klasy oraz prywatne elementy wyglądają tak jak na listingu nr.~\ref{lst:matrixclass}.

\begin{lstlisting}[caption=Klasa \texttt{Matrix}, label={lst:matrixclass}, language=C++]
	template <typename T>
	class Matrix {
		private:
		std::vector<std::vector<T>> data;
		
		public:
		
		Matrix(void) {}
		
		Matrix(int n) : data(n, std::vector<T>(n)) {}
		
		Matrix(int n, int* t) : data(n, std::vector<T>(n)) {
			for(auto& row : data){
				row = {t, t + n};
				t += n;
			}
		}
		
		Matrix(Matrix const& m) : data(m.data) {}
		
		void wypisz(void) {
			for(const auto& row : data){
				for(const auto& elem : row) {
					std::print("{} ", elem);
				}
				std::println();
			}
		}
		
		Matrix& alokuj(int n){
			if(data.size() == 0){
				data.resize(n, std::vector<T>(n));
			} else {
				if(data.size() < n){
					data.resize(n, std::vector<T>(n));
				}
			}
			return *this;
		}
		
		int pokaz(int x, int y) const {
			return data[x][y];
		}
		
		Matrix& dowroc(void) {
			Matrix<T> temp(data.size());
			
			for(int i = 0; i < data.size(); i++){
				for(int j = 0; j < data.size(); j++){
					temp.data[i][j] = data[j][i];
				}
			}
			
			*this = temp;
			return *this;
		}
		
		Matrix& losuj(void) {
			std::random_device rd;
			std::mt19937 gen(rd());
			std::uniform_int_distribution dis(0, 9);
			
			for(int i = 0; i < data.size(); i++){
				for(int j = 0; j < data.size(); j++){
					data[i][j] = dis(gen);
				}
			}
			return *this;
		}
		
		Matrix& losuj(int x) {
			if (data.size() == 0) return *this;
			
			std::random_device rd;
			std::mt19937 gen(rd());
			std::uniform_int_distribution<> dis(0, 9);
			
			int n = (int)data.size();
			for (int i = 0; i < x; i++) {
				int a = dis(gen) % n;
				int b = dis(gen) % n; // j.w.
				data[a][b] = dis(gen);
			}
			return *this;
		}
		
		Matrix& diagonalna(const int* t) {
			for(int i = 0; i < data.size(); i++){
				for(int j = 0; j < data.size(); j++){
					if(i == j){
						data[i][j] = t[i];
					} else {
						data[i][j] = 0;
					}
				}
			}
			return *this;
		}
		
		Matrix& diagonalna_k(int k, const int* t) {
			for(int i = 0; i < data.size(); i++){
				for(int j = 0; j < data.size(); j++){
					if(i == j + k){
						data[i][j] = t[j];
					} else {
						data[i][j] = 0;
					}
				}
			}
			return *this;
		}
		
		Matrix& kolumna(int x, const int* t) {
			for(int i = 0; i < data.size(); i++){
				data[i][x] = t[i];
			}
			return *this;
		}
		
		Matrix& wiersz(int x, const int* t) {
			for(int i = 0; i < data.size(); i++){
				data[x][i] = t[i];
			}
			return *this;
		}
		
		Matrix& przekatna(void) {
			for(int i = 0; i < data.size(); i++){
				for(int j = 0; j < data.size(); j++){
					if(i == j){
						data[i][j] = 1;
					} else {
						data[i][j] = 0;
					}
				}
			}
			return *this;
		}
		
		Matrix& pod_przekatna(void) {
			for(int i = 0; i < data.size(); i++){
				for(int j = 0; j < data.size(); j++){
					if(i > j){
						data[i][j] = 1;
					} else {
						data[i][j] = 0;
					}
				}
			}
			return *this;
		}
		
		Matrix& nad_przekatna(void) {
			for(int i = 0; i < data.size(); i++){
				for(int j = 0; j < data.size(); j++){
					if(i < j){
						data[i][j] = 1;
					} else {
						data[i][j] = 0;
					}
				}
			}
			return *this;
		}
		
		Matrix& wstaw(int x, int y, int wartosc) {
			data[x][y] = wartosc;
			return *this;
		}
		
		Matrix& szachownica(void) {
			for (int i = 0; i < (int)data.size(); i++) {
				for (int j = 0; j < (int)data.size(); j++) {
					data[i][j] = ((i + j) % 2 == 0) ? 0 : 1;
				}
			}
			return *this;
		}
		
		Matrix& operator+(Matrix const& m) {
			if (data.size() != m.data.size()) {
				return *this;
			}
			
			for (int i = 0; i < (int)data.size(); i++) {
				for (int j = 0; j < (int)data.size(); j++) {
					data[i][j] += m.data[i][j];
				}
			}
			return *this;
		}
		
		Matrix& operator*(Matrix const& m) {
			int n = (int)data.size();
			if (n == 0 || n != (int)m.data.size()) {
				return *this;
			}
			
			Matrix<T> result(n);
			for (int i = 0; i < n; i++) {
				for (int j = 0; j < n; j++) {
					T sum = 0;
					for (int k = 0; k < n; k++) {
						sum += data[i][k] * m.data[k][j];
					}
					result.data[i][j] = sum;
				}
			}
			*this = result;
			return *this;
		}
		
		Matrix& operator+(int a) {
			for (auto& row : data) {
				for (auto& elem : row) {
					elem += a;
				}
			}
			return *this;
		}
		
		Matrix& operator*(int a) {
			for (auto& row : data) {
				for (auto& elem : row) {
					elem *= a;
				}
			}
			return *this;
		}
		
		Matrix& operator-(int a) {
			for (auto& row : data) {
				for (auto& elem : row) {
					elem -= a;
				}
			}
			return *this;
		}
		
		friend Matrix operator+(int a, Matrix const& m) {
			Matrix result(m);
			for (auto& row : result.data) {
				for (auto& elem : row) {
					elem = a + elem;
				}
			}
			return result;
		}
		
		friend Matrix operator*(int a, Matrix const& m) {
			Matrix result(m);
			for (auto& row : result.data) {
				for (auto& elem : row) {
					elem = a * elem;
				}
			}
			return result;
		}
		
		friend Matrix operator-(int a, Matrix const& m) {
			Matrix result(m);
			for (auto& row : result.data) {
				for (auto& elem : row) {
					elem = a - elem;
				}
			}
			return result;
		}
		
		Matrix& operator++(int) {
			for (auto& row : data) {
				for (auto& elem : row) {
					elem += 1;
				}
			}
			return *this;
		}
		
		Matrix& operator--(int) {
			for (auto& row : data) {
				for (auto& elem : row) {
					elem -= 1;
				}
			}
			return *this;
		}
		
		Matrix& operator+=(int a) {
			for (auto& row : data) {
				for (auto& elem : row) {
					elem += a;
				}
			}
			return *this;
		}
		
		Matrix& operator-=(int a) {
			for (auto& row : data) {
				for (auto& elem : row) {
					elem -= a;
				}
			}
			return *this;
		}
		
		Matrix& operator*=(int a) {
			for (auto& row : data) {
				for (auto& elem : row) {
					elem *= a;
				}
			}
			return *this;
		}
		
		Matrix& operator()(double a) {
			int val = (int)std::floor(a);
			for (auto& row : data) {
				for (auto& elem : row) {
					elem += val;
				}
			}
			return *this;
		}
		
		friend std::ostream& operator<<(std::ostream& o, Matrix const& m) {
			for (const auto& row : m.data) {
				for (const auto& elem : row) {
					o << elem << " ";
				}
				o << "\n";
			}
			return o;
		}
		
		bool operator==(const Matrix& m) const {
			if (data.size() != m.data.size() || data[0].size() != m.data[0].size()) return false;
			for (int i = 0; i < (int)data.size(); i++) {
				for (int j = 0; j < (int)data[i].size(); j++) {
					if (data[i][j] != m.data[i][j]) return false;
				}
			}
			return true;
		}
		
		bool operator>(const Matrix& m) const {
			if (data.size() != m.data.size() || data[0].size() != m.data[0].size()) return false;
			for (int i = 0; i < (int)data.size(); i++) {
				for (int j = 0; j < (int)data[i].size(); j++) {
					if (!(data[i][j] > m.data[i][j])) return false;
				}
			}
			return true;
		}
		
		bool operator<(const Matrix& m) const {
			if (data.size() != m.data.size() || data[0].size() != m.data[0].size()) return false;
			for (int i = 0; i < (int)data.size(); i++) {
				for (int j = 0; j < (int)data[i].size(); j++) {
					if (!(data[i][j] < m.data[i][j])) return false;
				}
			}
			return true;
		}
	};
\end{lstlisting}
  
Klasa posiada tyko jedno prywatne pole, którym jest \texttt{std::vector<std::vector<T>> data}. oraz publiczne metody takie jak:
\begin{itemize}
	\item Konstruktor domyslny
	\item Konstruktor z rozmiarem macierzy
	\item 3 metody podstawowe, takie jak wypisz, alokuj, pokaz.
	\item Pozostałe metody
\end{itemize}
Metody zostaną opisane poniżej.

\begin{itemize}
	\item \texttt{Konstruktor domyślny}\\
	Domyślny konstruktor, nie wykonuje żadnych działań poza utworzeniem obiektu.
	
	\item \texttt{Konstruktor z rozmiarem}\\
	Znajduje się w wierszu 10. Jest odpowiedzialny za utworzenie macierzy n*n wypełnionej wartościami domyślnymi T data() alokuje pamięć dla wektora o długości n, gdzie każdy element to wektor o długości n.
	
	\item \texttt{Konstruktor z tablicą}\\
	Znajduje się w wierszach od 12 do 17. Jest odpowiedzialny za tworzenie macierzy o n*n wypełnionej wartościami t. Instrukcja row kopiuje n elementów z tablicy t do bieżącego wiersza row. Instrukcja t+=n przesuwa wskaźnik do następnego zestawu elementów.
	
	\item \texttt{Konstruktor kopiujący}\\
	Znajduje się w wierszu 19. Jest odpowiedzialny za tworzenie kopii macierzy n poprzez skopiowanie jej danych.
	
    \item \texttt{wypisz()}\\
    Znajduje się w wierszach od 21 do 28.
    Jest odpowiedzialna za wyświetlanie elementów macierzy w konsoli. Pętla zewnętrzna jest odpowiedzialna za przechodzenie przez wiersze macierzy a wewnętrzna przez elementy wiersza.

    \item \texttt{alokuj()}\\
    Znajduje się w wierszach od 30 do 39.
    Jest odpowiedzialna za alokowanie pamięci dla macierzy o rozmiarze n*n. Jeżeli macierz jest pusta to tworzy nową macierz, jeżeli jest mniejsza od n to zmienia jej rozmiar na n*n.

    \item \texttt{pokaz()}\\
    Znajduje się w wierszach od 41 do 43.
    Zwraca wartość elementu na pozycji x,y w macierzy.

    \item \texttt{dowroc()}\\
    Znajduje się w wierszach od 45 do 56.
    Jest odpowiedzialna za zamianę wierszy z kolumnami. Tworzy tymczasową macierz \texttt{temp} aby wypełnić ją wartościami transponowanymi względem macierzy wejściowej. Następnie zastępuje bieżącą macierz nowymi wartościami.

    \item \texttt{losuj()}\\
    Znajduje się w wierszach od 58 do 69.
    Tworzy generator liczb pseudolosowych od 0 do 9 a następnie przy pomocy pętli for wypełnia całą macierz losowymi wartościami z zakresu.

    \item \texttt{losuj\_x\_elementów()}\\
    Znajduje się w wierszach od 71 do 85.
    Zasada działania jest podobna do poprzedniej metody losuj z zakresu ale tutaj losujemy x elementów.

    \item \texttt{diagonalna()}\\
    Znajduje się w wierszach od 87 do 98.
    Tworzy wzór szachownicy, gdzie pola mają wartość albo 0 albo 1, dzieje się to przy pomocy warunku (i+j) modulo 2 == 0.

    \item \texttt{diagonalna\_k()}\\
    Znajduje się w wierszach od 100 do 111,
    Jest odpowiedzialna za ustawienie wartości przekątnej przesuniętej o k miejsc.
    
    \item \texttt{kolumna()}\\
    Znajduje się w wierszach od 113 do 118.
    Jest odpowiedzialna za wypełnienie kolumny k wartościami z tablicy t.
    
    \item \texttt{wiersz()}\\
    Znajduje się w wierszach od 120 do 125.
    Jest odpowiedzialna za wypełnienie wiersza x wartościami z tablicy t.
    
    \item \texttt{przekatna()}\\
    Znajduje się w wierszach od 127 do 138.
    Jest odpowiedzialna za wypełnienie głównej przekątnej 1 a resztę 0.
    
    \item \texttt{pod\_przekatna()}\\
    Znajduje się w wierszach od 140 do 151.
    Jest odpowiedzialna za wypełnienie obszaru pod główną przekątną 1 a resztę 0.
    
    \item \texttt{nad\_przekatna()}\\
    Znajduje się w wierszach od 153 do 164.
    Jest odpowiedzialna za wypełnienie obszaru nad główną przekątną 1 a resztę 0.
    
    \item \texttt{wstaw()}\\
    Znajduje się w wierszach od 166 do 169.
    Jest odpowiedzialna za wstawienie wartości do macierzy na pozycji (x,y).
    
    \item \texttt{Szachownica()}\\
    Znajduje się w wierszach od 171 do 178.Jest odpowiedzialna za utworzenie wzoru szachownicy w macierzy.
    
    \item \texttt{Metody firend operator+(Matirx a)}\\
    Znajduje się w wierszach od 180 do 191. Jest odpowiedzialna za dodawanie macierzy do macierzy element po elemencie.
    
    \item \texttt{operator*(Matrix a)}\\
    Znajduje się w wierszach od 193 do 211.
    Jest odpowiedzialna za mnożenie macierzy przez inną macierz.
    
    \item \texttt{operator+(int a)}\\
    Znajduje się w wierszach od 213 do 220.
    Jest odpowiedzialna za dodawanie do każdego elementu macierzy liczby całkowitej.
    
    \item \texttt{operaotr*(int a)}\\
    Znajduje się w wierszach od 222 do 229.
    Jest odpowiedzialna za mnożenie każdego elementu macierzy przez liczbę całkowitą.
    
    \item \texttt{operator-(int a)}\\
    Znajduje się w wierszach od 231 do 238.
    Jest odpowiedzialna za odjęcie od każdego elementu macierzy liczby całkowitej.
    
    \item \texttt{operator+(int a, const n)}\\
    Znajduje się w wierszach od 240 do 248.
    Jest odpowiedzialna za dodawanie liczb całkowitej do każdego elementu macierzy(postać a+n).
    
    \item \texttt{operator*(it a, const n)}\\
    Znajduje się w wierszach od 250 do 258.
    Jest odpowiedzialna za mnożenie każdego elementu macierzy przez liczbę całkowitą(postać a*n).
    
    \item \texttt{operator-(int a, const n	)}
    Znajduje się w wierszach od 260 do 268.
    Jest odpowiedzialna za odejmowanie każdego elementu macierzy od liczby a(postać a-n).
    
    \item \texttt{operator++}\\
    Znajduje się w wierszach od 270 do 277.
    Jest odpowiedzialny za zwiększenie każdego elementu o 1.
    
    \item \texttt{operator--}\\
    Znajduje się w wierszach od 279 do 284.
    Jest odpowiedzialna za zmniejszenie wartości elementu o 1.
    
    \item \texttt{operator +=(int a)}\\
    Znajduje się w wierszach od 288 do 295.
    Dodaje liczbę całkowitą do kaźdego elementu macierzy.
    
    \item \texttt{operator -=(int a)}\\
    Znajduje się w wierszach od 297 do 304.
    Jest odpowiedzialna za odejmowanie od każdego elementu macierzy liczby całkowitej.
    
    \item \texttt{operator()(double a)}\\
    Znajduje się w wierszach od 315 do 323.
    Jest odpowiedzialna za dodawanie do każdego elementu macierzy cząstki liczby całkowitej
\end{itemize}
